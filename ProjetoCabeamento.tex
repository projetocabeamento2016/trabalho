%%% LaTeX Template: Two column article
%%%
%%% Source: http://www.howtotex.com/
%%% Feel free to distribute this template, but please keep to referal to http://www.howtotex.com/ here.
%%% Date: February 2011

%%% Preamble
\documentclass[	DIV=calc,%
							paper=a4,%
							fontsize=12pt,%
							onecolumn]{scrartcl}	 					% KOMA-article class

\usepackage{lipsum}													% Package to create dummy text
\usepackage[english]{babel}										% English language/hyphenation
\usepackage[protrusion=true,expansion=true]{microtype}				% Better typography
\usepackage{amsmath,amsfonts,amsthm}					% Math packages
\usepackage[pdftex]{graphicx}									% Enable pdflatex
\usepackage[svgnames]{xcolor}									% Enabling colors by their 'svgnames'
\usepackage[hang, small,labelfont=bf,up,textfont=it,up]{caption}	% Custom captions under/above floats
\usepackage{epstopdf}												% Converts .eps to .pdf
\usepackage{subfig}													% Subfigures
\usepackage{booktabs}												% Nicer tables
\usepackage{fix-cm}													% Custom fontsizes
\usepackage[utf8]{inputenc}
\usepackage[top=2.5cm, bottom=2.5cm, left=2.5cm, right=2.5cm]{geometry}
\usepackage[ddmmyyyy]{datetime}
\addto\captionsenglish{%
	\renewcommand\tablename{Tabela}
	\renewcommand\figurename{Figura}
} 
 

 
%%% Custom sectioning (sectsty package)
\usepackage{sectsty}													% Custom sectioning (see below)
\allsectionsfont{%															% Change font of al section commands
	\usefont{OT1}{phv}{b}{n}%										% bch-b-n: CharterBT-Bold font
	}

\sectionfont{%																% Change font of \section command
	\usefont{OT1}{phv}{b}{n}%										% bch-b-n: CharterBT-Bold font
	}



%%% Headers and footers
\usepackage{fancyhdr}												% Needed to define custom headers/footers
	\pagestyle{fancy}														% Enabling the custom headers/footers
\usepackage{lastpage}	

% Header (empty)
\lhead{}
\chead{}
\rhead{}
% Footer (you may change this to your own needs)

%% ====================================
%% ====================================
%% mude o rodape  do projeto
%% ====================================
%% ====================================

\lfoot{\footnotesize \texttt{Cabeamento estruturado.} \textbullet ~Projeto Bloco "K" UTFPR/CP}


\cfoot{}
\rfoot{\footnotesize página \thepage\ de \pageref{LastPage}}	% "Page 1 of 2"
\renewcommand{\headrulewidth}{0.0pt}
\renewcommand{\footrulewidth}{0.4pt}



%%% Creating an initial of the very first character of the content
\usepackage{lettrine}
\newcommand{\initial}[1]{%
     \lettrine[lines=3,lhang=0.3,nindent=0em]{
     				\color{DarkGoldenrod}
     				{\textsf{#1}}}{}}



%%% Title, author and date metadata
\usepackage{titling}															% For custom titles

\newcommand{\HorRule}{\color{DarkGoldenrod}%			% Creating a horizontal rule
									  	\rule{\linewidth}{1pt}%
										}

\pretitle{\vspace{-30pt} \begin{flushleft} \HorRule 
				\fontsize{50}{50} \usefont{OT1}{phv}{b}{n} \color{DarkRed} \selectfont 
				}

%% ====================================
%% ====================================
%% mude o titulo  do projeto
%% ====================================
%% ====================================

\title{Projeto Cabeamento Estruturado}					% Title of your article goes here

%% ====================================



\posttitle{\par\end{flushleft}\vskip 0.5em}

\preauthor{\begin{flushleft}
					\large \lineskip 0.5em \usefont{OT1}{phv}{b}{sl} \color{DarkRed}}
\author{Pedro Holtz, Marcelo Machado, Filipe Bonacin, Ademir, Wellington, Jozua Henrique }  	% Author name goes here


\postauthor{\footnotesize \usefont{OT1}{phv}{m}{sl} \color{Black} 
					\\Universidade Tecnológica Federal do Paraná - Câmpus Cornélio Procópio 								% Institution of author
					\par\end{flushleft}\HorRule}

\date{}																				% No date




%%% Begin document
\begin{document}
\maketitle
\thispagestyle{fancy} 	
\thispagestyle{empty}		% Enabling the custom headers/footers for the first page 
% The first character should be within \initial{}




%% ====================================
%% ====================================
%% mude o resumo  do projeto
%% ====================================
%% ====================================
\initial{E}\textbf{ste modelo exemplifica como um projeto de cabeamento estruturado deve ser elaborado. No projeto real,
	esta seção deve indicar qual seria o propósito do projeto: reestruturar, criar uma estrura sem uma anterior, apresentar uma estrutura ficticia ou real  etc. Deve também conter sua cobertura (quais elementos este projeto aborda): Levantamento da planta física, Elaboração da planta lógica, Equipamentos passivos da rede (memorial descritivo), Levantamento de quantidade/custo, Plano de Certificação e orçamento.
	No caso de projetos reais, deve conter também o escopo. No projeto real deve conter quais as atividades serão executadas e quais os responsáveis. Deve apresentar também as restrições que serão cumpridas e um possível acordo de nível de serviço (SLA).}

%% ====================================
\begin{figure}
	\centering
	\includegraphics{utfpr}
\end{figure}

\vspace{3cm}
\centerline{\textit{\textbf{\today}}}

\clearpage
    \renewcommand*\listfigurename{Lista de figuras}
\listoffigures

\renewcommand*\listtablename{Lista de tabelas}
\listoftables




\clearpage
\renewcommand{\contentsname}{Sumário}
\tableofcontents
\clearpage

%% ====================================
%% ====================================
%% Inicio do texto
%% ====================================
%% ====================================
\section{Introdução}
Explique nesta primeira seção qual seria o perfil do caso. Perfil do cliente, quantidade de colaboradores, quantidade de equipamentos de TI atualmente.

Indique também nesta seção o escopo do projeto.

Apresente um overview do parque tecnológico do caso.
\subsection{Benefícios}
\begin{itemize}
\item Gerenciamento mais eficiente dos ativos de rede.
\item Rapidez e facilidade na identificação de problemas na camada física do modelo OSI.
\item Diminuição nos custos de mão de obra e montagem de infra-estrutura. 
\item Substituição rápida de ativos de rede quando preciso, devido a ordenação dos cabos. 
\item Documentação técnica para que qualquer profissional, não necessariamente o que atuou na estruturação inicial, possa fazer novas implantações ou alterações. 
\item Localização fácil de um cabo, devido à identificação em todo o sistema. Facilidade na manutenção de uma área/estação de trabalho.
\end{itemize}

\section{Estado atual}
Aprente o estado atual da rede. Caso não tenha rede, desconsiderar esta seção.

Caso tenha rede, deixe claro:
\begin{itemize}
	\item os passivos de rede atuais:path panels, cabos, etc..;
	\item as principais reclamações dos usuários. Qual o principal motivo da reestruturação? Efetue uma pesquisa junto aos colaboradores para determinar quais problemas a rede apresenta.
	\item Observações. Analise a rede e verifique se há estruturas que não se enquadram nas normas ou que indicam suspeita de problemas.
\end{itemize}


\section{Usuários e Aplicativos}
Explique nesta seção os usuários atuais e o perfil de crescimento, se por exemplo, há estimativa na evolução da empresa no que tange a quantidade de usuários, pontos de redes, equipamentos.
 

\subsection{Usuários}
Crie uma relação da quantidade, perfil de usuários de seu projeto.

\subsection{Aplicativos}
Crie uma relação dos aplicativos e seus níveis críticos de uso.


\section{Estrutura predial existente}

Explique aqui a planta física dos prédios
Pode ser anexada, em escala ou não.

Deve conter uma descrição geral, indicando a possível distância entre os pontos de rede e restrições de instalação.

\section{Planta Lógica - Elementos estruturados}

\subsection{Estado atual}
Deve ter a planta atual, se for o caso

\subsection{Topologia}
Proposta futura, proposta após implantação.
Deve conter o diagrama da rede. Atente-se a redundância  e ligações truncadas.
Deve explicar todos termos e componentes utilizados nestas plantas. Por exemplo: entrance facility, work area, horizontal cabling, etc..

Todos os elementos das figuras devem ser explicados. 
Crie esboço da configuração dos racks e brackets. Explique cada um dos componentes. Você pode criar uma tabela contendo figuras dentro, ou criar uma tabela e incluí-la como imagem. Por exemplo, verifique a tabela \ref{tab1}.

\input{tab1}

\subsection{Encaminhamento}
Eletrodutos, calhas, e qualquer material em que os cabos serão alojados/alocados.

\subsection{Memorial descritivo}

Relacione todos os equipamentos passivos que serão utilizados, tipo, fabricante, quantidade.

\subsection{Identificação dos cabos}

\section{Implantação}
Estabeleça um cronograma de implantação:
Remoção de equipamentos existentes (destino para descarte), instalação dos condutores, instalação dos cabos, 
identificação dos cabos, montagem dos racks, certificação, etc... Crie atividades e estabeleça o tempo de execução. Se for um projeto real, indique também quais os responsáveis pela execução do projeto e de cada uma das etapas.

Defina marcas (e padrões) e fornecedores se for o caso. Atenção a contratados e subcontratados para a realização das atividades. Estabeleça a responsabilidade de execução da atividade e também da validação dela.

Utilize algum software para gerear o cronograma. Excel,etc. O fundamental é dividir em etapas, descrever e estimar o tempo de cada uma delas.

Segue uma relação de ferramentas:
http://asana.com/, 
https://trello.com/, 
http://www.ganttproject.biz/, 
http://www.orangescrum.org/. 

\section{Plano de certificação}
A baixo foi relacionado as etapas seguidas para a certificação:
\begin{itemize}
\item Paradiafonia (NEXT)
\item Verifica a quantidade de conexões no link;
\item Impedância do cabo - Expressa a contribuição das resistências, indutâncias, capacitâncias e condutâncias distribuídas ao longo do condutor, e medida em campo por meio de cable scanners. A qualidade de construção do cabo, é principal determinante no valor da impedância do mesmo.
\item Atenuação do cabo - Perda de potência do sinal transmitido – quanto maior a frequência do sinal pior é o caso (efeito skin ).
\item ACR (atenuação x NEXT)- Importante parâmetro a ser medido que expressa relação entre a Atenuação e o NEXT .
\item Return Loss (perda de retorno) - Reflexões causadas por anomalias na impedância característica ao longo de um segmento de cabo.
\end{itemize}


\section{Plano de manutenção}

Por meio dos serviços de analise e diagnóstico de rede, é realizado um trabalho forense de cada dispositivo na rede por criticidade de sua operação que permite diagnosticar os gargalos e sugerir ações práticas de correção.
Será realizado trimestralmente a manutenção e execução de serviços de analise e diagnóstico de rede. Desta forma, é possível garantir elevado nível de serviço exigido pela rede para atender o tráfego de voz, imagem e outros dados.
Quando necessário adicionar um novo ponto de rede, deverá respeitar as normas utilizadas no projeto. Após a adição de um novo ponto de rede, se faz necessário realizar teses conforme a certificação utilizada no projeto. Assim, é possível garantir que tudo após o serviço a rede continua funcionando de forma esperada.
O propósito de um sistema de cabeamento estruturado é garantir uma base sólida para o bom desempenho das redes de comunicação de voz, imagem e outros dados devem permitir mudanças e alterações de layout nas demandas de mudança.
\subsection{Manutenção corretiva}
Os procedimentos acima contribuem para viabilizar a manutenção corretiva, que é aquela de atendimento imediato para consertar equipamentos danificados ou que sofreram avarias. Normalmente, o número de avarias cresce à medida que não são tomadas medidas antecipadas para o perfeito funcionamento dos equipamentos.
Este tipo de manutenção é considerado como um dos que mais onera a produção, porque, normalmente, tal manutenção implica na parada do equipamento e interrupção da produção. Por isso, a equipe de manutenção deve trabalhar com eficácia para evitar que os equipamentos sempre parem precisando de manutenção corretiva.
\subsection{Manutenção preventiva}
O treinamento da equipe de manutenção deve ser contínuo, pois tal procedimento é indispensável para garantir maior disponibilidade e confiabilidade dos equipamentos existentes na rede. Para um efetivo controle da manutenção preventiva é necessário monitorar o tráfego de rede e analisar o desempenho no que tange as transmissões por meio físico da rede. 
\subsection{Equipe de suporte}
A equipe de suporte de redes, terão que estar preparados para resolver possíveis problemas que possam ocorrer durante as atividades dos funcionários e alunos da UTFPR/CP. Caso ocorra uma ocorrência em um ponto de rede, o suporte técnico deve identificar o local onde ocorreu o problema, e o mais rápido possível a equipe de suporte se deslocar até o local afetado, analisar o problema e resolvê-lo.


\subsection{Plano de expansão}
Existe um plano de expansão? Quantos novos pontos poderão ser acrecidos na rede, antes de migração de equipamentos na camada 2? Se houver expansão, quais equipamentos deverão ser direcionados para as estremidades da rede? 


\section{Orçamento}
Crie uma relação de orçamentos baseado na seções anteriores.

\section{Referências bibliográficas}
Utilize o mendley, o jabref ou diretamente o bibtex para gerenciar suas referências biliográficas. As referências são criadas automaticamente de acordo com o uso no texto.

Exemplo: Redes de computadores, segundo \cite{t2013} é considerada..... Já \cite{kurose2010} apresenta uma versão...

Analisando os pressupostos de \cite{ref3} e \cite{ref4} concluimos que....


\renewcommand\refname{} %%Referências bibliográficas}  
\bibliographystyle{ieeetr}
\bibliography{referencias}  

%% ***********************************************************************
%% === remover daqui =====================================================
%% ***********************************************************************

\section{Elementos textuais - Alguns exemplos}

Esta seção apresenta exemplos de elementos textuais. \textbf{Remova-a da versão final do texto}.


\subsection{Colocar elementos em itens}

Texto antes da lista

\begin{itemize}
	\item First item in a list 
	\item Second item in a list 
	\item Third item in a list
\end{itemize}

\subsubsection{Uma sub seçao de terceiro nivel}

Exemplo de uma subseção

\subsection{Tabelas}

Utilize o site http://www.tablesgenerator.com/ para elaborar as tabelas de seu trabalho.
Para adicionar uma tabela utilize: a tag input, passando o arquivo da tabela como parametro

\input{tab2}

Dentro do arquivo você deve definir o label e pode utilizá-lo para referenciar. Exemplo:
Na tab \ref{tab2} temos a relação de ....


Você também pode modificar a tabela manualmente, incluindo, por exemplo h! dentro de sua definição. Veja no exemplo tab2.tex

\subsection{Figuras}

As figuras podem ser no formato PDF, JPG, PNG. Você pode referenciá-las da mesma maneira que tabelas. Exemplo: A figura \ref{fig1} apresenta.....

Não se preocupe o local em que a figura será renderizada em seu texto. Preocupe-se em criar referência para ela, ou seja, toda figura e tabela deve conter pelo menos uma referência no texto.

\begin{figure}
\centering
\includegraphics[width=\textwidth]{fig1}
\caption{Exemplo de figura com escala horizontal}
\label{fig1}
\end{figure}


\begin{figure}
	\centering
	\includegraphics[]{fig2}
	\caption{Exemplo de figura sem escala}
	\label{fig2}
\end{figure}

Você pode rotacionar figuras também. Para isso utilize o parâmetro angle=-90. Repare que a escala da figura foi modificada pelo parametro height. Você também pode utilizar scale

\begin{figure}
	\centering
	\includegraphics[height=\textwidth,angle=-90]{fig3}
	\caption{Exemplo de figura rotacionada}
	\label{fig3}
\end{figure}


%% ***********************************************************************
%% === ate aqui    =====  ================================================
%% ***********************************************************************
\end{document}